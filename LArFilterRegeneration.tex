\documentclass[letterpaper,11pt]{article}
\usepackage[margin=1in]{geometry}
\usepackage{natbib}
% \bibliographystyle{unsrtnat}
\usepackage{tabularx} % extra features for tabular environment
\usepackage{amsmath}  % improve math presentation
\usepackage{graphicx} % takes care of graphic including machinery
% \usepackage{cite} % takes care of citations
\usepackage[final]{hyperref} % adds hyper links inside the generated pdf file
\usepackage{longtable}

% ------------------------------------------------------------------------
\newcommand{\myCheckBox}{\CheckBox[width=0.8em,bordercolor={0.65 0.79 0.94},height=0.8em]}
\renewcommand{\arraystretch}{1.3}

% ------------------------------------------------------------------------

\begin{document}

\title{\textbf{LAr Filter Regeneration Procedure}}
\author{Yun-Tse Tsai}
\date{\today}

\maketitle

%-------------------------------------------------------------------

Notes:
\begin{itemize}
\setlength\itemsep{-0.2em}
\item P\&ID is attached in the end of this list
\item The weight for the all metal valve, V9, is 4 pounds
\item The tube upstream V9 is not leak tight (mainly the connections to the flowmeter).  Therefore keep V9 closed when we need to isolate the LAr filter
\item The gas flow has to be greater than 130~scfm (marked on the flowmeter) to prevent the heater from getting too hot
\item We should keep the gas flow (Ar or 2\%H$_2$+Ar) between 130 and 400~scfm (label on the flowmeter), but preferably at 400~scfm (the upper range of the flowmeter)
\item Maintain the catalyst temperature between about $170^\circ$C and
$225^\circ$C
\item Do NOT exceed $225^\circ$C, even though $250^\circ$C may be tolerated
\item If we keep the gas flow at 400~scfm, a gas bottle should be finished in about an hour
\item If seeing smoke or smelling something unusual, shut down the variac power supply (heater) and investigate.

\end{itemize}
%-------------------------------------------------------------------

\tabcolsep=10pt
\begin{longtable}{p{0.5\textwidth}p{0.5\textwidth}}
\hline
\hline
Checklist & What to Do and Detailed Description \\
\hline
\multicolumn{2}{l}{\textbf{Preparation}} \\
\myCheckBox{3 bottle of ultra high purity Ar gas (TBC)} & \\
\myCheckBox{5 bottles of Ar+2\%H$_2$ gas (TBC)} & \\
\myCheckBox{Tubes connecting the heater and the LAr filter wrapped with aluminum foils for thermal insulation} & \\
\myCheckBox{V4 connected to the scroll pump} & Prepare to evacuate the vacuum vessel \\
\myCheckBox{V4 open, scroll pump on} & Evacuate the vacuum vessel \\
\myCheckBox{V3, V5, V6, V7, V8, V9, V10, V11, V12 closed} & \\
\myCheckBox{Exhausting gas line connected and humidity meter hooked} & \\

\hline
\multicolumn{2}{l}{\textbf{Preheating with Ar gas}} \\
\myCheckBox{V3, V5, V6, V7, V8, V9, V10, V11, V12 closed} & \\
\myCheckBox{PG6 at 0 psi} & \\
\myCheckBox{Variac power supply off.  Voltage set at 0} & \\
\myCheckBox{Heater plugged in to the variac power supply} & \\
\myCheckBox{Ar gas bottle connected to Reg1 and V7/V8 line} & \\
\myCheckBox{GMV1 opened, Reg1 increased, V7 opened, air purged} & Purge the air in the connection tube \\
\myCheckBox{V7 closed} & Finish purging \\
\myCheckBox{V8, V9, V6 opened} & \\
\myCheckBox{PG3 $>$~2 psig, V10 opened} & \\
\myCheckBox{Gas flow $>$~130~scfh, stable} & \\
\myCheckBox{Variac power supply on, increase the voltage} & Turn on the heater \\
\myCheckBox{Humidity plateaued for $>$~10~minutes} & Molecular sieves regenerated \\
\myCheckBox{Preheated for $>$~2~hours} & \\
\myCheckBox{Variac power supply off.  Voltage set at 0} & Turn off the heater \\
\myCheckBox{V8, V9, V10 closed} & \\
\myCheckBox{GMV1 and Reg1 closed} & \\

\hline
\multicolumn{2}{l}{\textbf{Regenerating copper sieves}} \\
\myCheckBox{Ar+2\%H$_2$ gas bottle connected to Reg1 and V7/V8 line} & \\
\myCheckBox{GMV1 opened, Reg1 increased, V7 opened, air purged} & Purge the air in the connection tube \\
\myCheckBox{V7 closed} & Finish purging \\
\myCheckBox{V8, V9, V6 opened} & \\
\myCheckBox{PG3 $>$~2 psig, V10 opened} & \\
\myCheckBox{Gas flow between 50 and 160~slpm (Ar), or between 130 and 400~scfh (labeled as Air).  Preferrably at 400~scfh Air} & \\
\myCheckBox{Variac power supply on, increase the voltage} & Turn on the heater \\
\myCheckBox{Should the temperature exceed $225^\circ$C anywhere in the bed, switch to H$_2$-free gas until the hot zone cools back down to 200 -- $210^\circ$C, then resume feeding the H$_2$ gas mixture} & \\
\myCheckBox{The temperature of the all catalyst bed is stable or subsiding} & \\
\myCheckBox{Humidity plateaued for $>$~10~minutes} & Copper sieves regenerated \\
\myCheckBox{Variac power supply off.  Voltage set at 0} & Turn off the heater \\
\myCheckBox{V6, V8, V9, V10 closed} & \\
\myCheckBox{GMV1 and Reg1 closed} & \\
\myCheckBox{Humidity meter unmounted} & \\


\hline
\multicolumn{2}{l}{\textbf{Completion}} \\
\myCheckBox{V3, V5, V6, V9 closed} & Prepare to evacuate the LAr filter \\
\myCheckBox{Scroll pump connected to V5} & \\
\myCheckBox{V5 open, scroll pump on} & \\
\myCheckBox{PG3 at 0~psig} & \\
\myCheckBox{V5 closed} & \\
\myCheckBox{V6, V10, V11, V12 closed} & Prepare to evacuate the piece of the plumbing system \\
\myCheckBox{Scroll pump connected to V10} & \\
\myCheckBox{V10 open, scroll pump on} & \\
\myCheckBox{Pumping 10 minutes} & \\
\myCheckBox{V10 closed} & \\
\myCheckBox{V3, V5, V9 closed} & \\
\myCheckBox{V17 closed, turbo pump off?} & \\
\myCheckBox{V12, V6 open?} & \\

\hline
\hline
\end{longtable}

\clearpage
\begin{figure}[htb]
\begin{center}
\includegraphics[angle=90,origin=c,height=7.5in]{/Users/yuntse/Documents/DUNE/SLArchetto/PID/PID_SLArchetto_20220406-Model.pdf}
\caption{P\&ID}
\end{center}
\end{figure}

\end{document}
